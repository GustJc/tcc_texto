\chapter*[Introdução]{Introdução}
\addcontentsline{toc}{chapter}{Introdução}


\section*{Introdução}

É cada vez mais visível em jogos a utilização de conteúdos proceduralmente gerados. A demanda por jogos cada vez mais rejogáveis aumentou muito nos últimos anos, o que é observado através da alta utilização de conteúdos procedurais em jogos modernos, com gráficos tanto de perspectiva tanto 2D quanto 3D. 

Não se pode esquecer da influência dos primeiros jogos que utilizaram este tipo de conteúdos automáticos. O gênero \textit{roguelike} surgiu em meados dos anos 80 pelo jogo \textit{Rogue}, que popularizou-se na época devido a seu aparente infinito número de possibilidades de jogabilidade, consequência do fato de que grande parte do seu conteúdo como mapas e ítens, serem randômicamente gerados, assim como seus gráficos representativos em ASCII(\textit{American Standard Code for Information Interchange}).

Com a evolução dos computadores e do gênero em si, jogos  \textit{roguelike} tiveram que evoluir também, apesar de ainda haver jogos recentes que preservam o aspecto clássico do gênero, com gráficos ainda em ASCII ou formados \textit{tiles} 2D. Existem também jogos do gênero com gráficos mais modernos e animações, em plataformas móveis, como \textit{Pokemon Mystery Dungeon}\footnote{Pokemon - \cite{chunsoft}} e \textit{Izuna}\footnote{Izuna - \cite{izuna}}, de Nintendo DS e em plataformas recentes como \textit{Guided Fate Paradox}\footnote{Guided Fate to Paradox - http://nisamerica.com/games/guided\_fate\_paradox}, para PS3. O que mostra que o gênero ainda tem público nos dias atuais.
Apesar de usarem tão fortemente conteúdos gerados proceduralmente, ainda hoje não é possível obter uma boa métrica de forma precisa quanto qualidade de seus mapas. Normalmente tendo a adequação dos seus mapas gerados através de muitos testes e apenas do sentimento obtido após muitas partidas.
 
 Não há nada de errado com esta forma, porém se houvesse uma ferramenta capaz de produzir métricas de análise para os mapas randômicamente gerados de forma relativamente rápida, isto poderia acelerar o processo de aperfeiçoamento dos algoritmos devido a redução em parte do tempo de testes. 
 
 Neste projeto, será desenvolvido um sistema capaz de obter um mapa e testa-lo através de uma série de métricas para assegurar a qualidade do mapa em relação a qualidade desejada. Na segunda parte do projeto será realizado uma analise comparativa para indicar se mapas proceduralmente gerados podem possuir métricas similares a mapas construídos manualmente e como usuários reais se sentem em relação as métricas obtidas, assegurando a confiança do sistema em seus resultados. 

\section*{Organização do documento}

O próximo capítulo, Objetivos, irá apresentar os objetivos gerais, representando o que o projeto se propõe a alcançar, e os objetivos específicos, mostrando o que será alcançado para resolver os objetivos gerais propostos. 

O Referencial Teórico irá mostrar alguns conceitos sobre a concepção e a história por traz do gênero de jogo \textit{roguelike}, assim como uma base teórica sobre conteúdos procedurais e taxonomias baseadas em artigos científicos e notícias.

A Justificativa irá explicar a importância do trabalho e o que o motivou a ser criado, demonstrando também seus benefícios para a sociedade no âmbito do desenvolvimento de jogos do gênero.

O quarto capítulo, Delimitação do Assunto, irá tratar de explicitar o que será desenvolvido e representado no trabalho e o que não entrará em seu escopo para que não haja dúvidas quanto suas expectativas. 

No quinto capítulo, Procedimentos Metodológicos e Técnicas, inicia-se o desenvolvimento, aonde serão demonstradas técnicas e procedimentos utilizados para a concepção e pesquisa do projeto, assim como a metodologia aplicada para a realização deles. 

O sexto capítulo, Cronograma, retratará como foi executado o cronograma do projeto durante esta sua primeira etapa e como será executada a segunda parte do projeto.

O ultimo capítulo, Resultados Alcançados, irá descrever que foi desenvolvido e obtido através da realização do trabalho durante está primeira etapa. 

