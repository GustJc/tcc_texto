\chapter[Resultados]{Resultados}

\section{Técnicas de coleta de metrica}



\section{Avaliação das métricas colhidas}



\chapter{Resultados Alcançados}

	A realização do trabalho até este ponto resultou em um protótipo de parte da ferramenta. O protótipo é capaz de carregar mapas, ítens e inimigos a partir do formato especificado e realizar uma simulação básica de jogos. 
	A ferramenta é capaz de coletar métricas básicas sobre:
\begin{itemize}			
	\item Quantidade de passos;
	\item Inimigos derrotados;
	\item Porcentagem de vitórias;
	\item Dinheiro coletado;
\end{itemize}

	Ainda não é realizada a concretização das métricas em um valor único de qualidade esperada, porém ela gera as médias, desvios padrões e variâncias para medidas de $N$ jogadas em um mesmo mapa. 
	
	O sistema possui um BOT parametrizado com variáveis de ganância com itens, \textit{tiles} solitários e número de iterações extras, podendo ser alterado somente através da modificação dentro do \textit{script} Lua no momento. Não há ainda a diferenciação de perfis por parte do sistema. 
	
	Ítens e inimigos são carregados através de \textit{scripts} Lua, atribuindo-os a um identificador de texto para serem chamados pela função de geração procedural do mapa de forma mais consistente. 
	
	A ferramenta disponibiliza uma simples interface de botões e menus para ajustes de parâmetros para o algoritmo procedural do mapa que será gerado ao iniciar o jogo. É disponibilizado também um botão para carregar um mapa de um arquivo, sem a geração de um novo mapa. 
	
	As devidas condições de vitória e derrota para o jogador estão implementadas, e o sistema possui um console interno que pode ser chamado para alterar propriedades de \textit{debug}, como \textit{setFog} para remoção ou adição de visibilidade, ou \textit{botDelay}, para ajustes na velocidade de processamento do BOT. 
	
	Inimigos estão atacando o jogador quando dentro de sua área de visão e as devidas mensagens de informações são mostradas em uma área na parte de cima da tela como: \lq\lq Você causou 5 de dano ao inimigo\rq\rq.
	
	O sistema está utilizando gráficos 16x16 pixels para representação visual de seus elementos. 
\section{Conclusão}
	O trabalho realizado possibilitou a confirmação da viabilidade da ferramenta e constatou que a necessidade que se tem em obter métricas concretas para mapas gerados automaticamente para acelerar o processo de desenvolvimento de jogos do gênero \textit{roguelike}, é um fator real.	
	
	A definição de diversos perfis de usuário foram observados e classificados para que uma melhor representação das métricas obtidas dos jogos simulados. 
	A ferramenta para a avaliação dos mapas, assim como a forma com que serão testados e extraídas as métricas foram estabelecidas, confirmando-as através da aplicação prática do sistema através de protótipos obtendo a comprovação de que seus conceitos básicos estão funcionando e sendo aplicados. 
	
	O trabalho obteve resultados positivos quanto aos seus benefícios para o desenvolvimento dos jogos, possibilitando a realização de simulações e obtenção de métricas de dezenas de jogos em apenas alguns minutos. 






%CITES
\begin{comment}
\nocite{ArTerrainGen}
\nocite{Decipher}
\nocite{DesAudVid}
\nocite{DesAud}
\nocite{Noel}
\nocite{TaxonomySearchBasePCG}
\nocite{Voronoid}
\nocite{automaticDunGen}
\nocite{contest}
\nocite{pcgwiki}
\nocite{chunsoft}
\nocite{ArTerrainGen}
\nocite{izuna}
\nocite{depthsearch}
\nocite{Astar}
\nocite{normal}
\nocite{montgomery}
\end{comment}