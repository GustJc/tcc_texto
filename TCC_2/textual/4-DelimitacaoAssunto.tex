\chapter[Delimitação do Assunto]{Delimitação do Assunto}

Aqui serão representados o que este trabalho propõe-se a desenvolver e quais resultados podem ser esperados através do uso da ferramenta a ser desenvolvida.

\section{Proposta}

Construir uma ferramenta capaz de medir a qualidade esperada de um mapa ou algoritmo procedural.
Propõe-se também a longo prazo, mostrar os benefícios que conteúdos procedurais trazem a jogos e uma comparação entre mapas feitos manualmente e automaticamente para demonstrar a diferença de qualidade entre os dois modelos de construção de mapas. 

\section{Ferramentas desenvolvidas}

A principal ferramenta desenvolvida auferirá a qualidade de um mapa carregado através de um arquivo externo com o seu devido formato. A partir de entradas sobre os valores esperados pelo usuário da ferramenta, ela irá gerar saídas sobre a qualidade do mapa em relação aqueles quesitos especificados e aos diversos perfis de usuário.

A ferramenta também terá suporte à carga de algoritmos de geração de mapas através da sua codificação de arquivos externos no formato \textit{lua}, evitando o desgaste de ter que formatar cada um dos mapas para o formato da ferramenta e testá-los um a um. 

Uma ferramenta auxiliar para a criação de mapas manuais e para criação e testes de algoritmos de criação procedurais em \textit{lua}. Contendo também um editor de \textit{scripts} com uma listagem de comandos e variáveis que podem ser entendidas pela ferramenta para se gerar o mapa através dele.

\subsection{Especificações}
\begin{itemize}
\item O sistema deve ser capaz de obter um valor desejado para cada uma das métricas que estará testando a fim de garantir que o mapa estará a qualidade desejada para aquele propósito.
\item Deverá permitir que seja dado um peso para cada métrica, aferindo peso 0 para métricas indesejadas.
\item Ler um arquivo externo, de acordo com a formatação pré-estabelecida e carregar a partir dele um mapa com o posicionamento de inimigos e itens.
\item A ferramenta deverá ser capaz de simular jogos, baseados em regras simples do gênero, para testar os mapas.
\item Deverá analisar diversos jogos utilizando uma IA customizada para simular os diversos perfis de usuários encontrados.
\item Extrair métricas de diversos jogos simulados e compilá-las em um parâmetro de qualidade, baseado nos valores esperados.
\item Deverá ser capaz carregar algoritmos de geração mapas escritos em \textit{lua} para que possam ser testado a qualidade do algoritmo como um todo  a partir dos mapas gerados por ele.
\item Deverá entender a padronização de ítens e inimigos em arquivos lua separados, para que possa escrever a criação dos mapas de forma mais clara e limpa.
\end{itemize}



\subsection{Escopo}

Está dentro do escopo a criação de uma ferramenta capaz de simular regras básicas de um jogo e extrair métricas a partir dele. A criação de mapas baseados em arquivos ou algoritmos externos, dado que estejam no devido formato entendido pelo sistema. E a obtenção de uma métrica de qualidade para um devido mapa ou algoritmo testado.

\subsection{Não Escopo}

Não está no escopo deste projeto desenvolver um jogo completo, ou qualquer tipo de jogo. A simulação representado pela ferramenta, apesar de ser jogável, contém apenas os requisitos básicos para se testar uma amostra genérica dos atributos contidos em um jogo do gênero.

Não está no escopo a criação de algoritmos procedurais novos e otimizados, apesar de que serão utilizados algoritmos procedurais para gerar as entradas da ferramenta. Estes não serão completamente inventados ou otimizados e representarão diferentes tipos de geração procedural para dar a diversidade ao se testar a ferramenta e garantir que ela funciona apesar da diferença estrutural entre no que diz respeito à mapas.

Não está no escopo a criação de algoritmos de inteligência artificial completamente inéditos. Algoritmos de caminhos como \textit{A*} ou \textit{Depth First Search} serão utilizados com ligeiras alterações para se adaptarem ao sistema. 

\subsection{\textit{Meta-Engine} - Ferramenta principal}

A principal ferramenta do sistema é responsável por rodar simulações sobre um dado mapa e aferir a qualidade do mesmo através de uma série de parâmetros definidos pelo usuário. Ela possui uma interface de comandos shell chamada pela tecla ';' que permite a execução de comandos simples do sistema como para alterar a velocidade de processamento da IA. 

Esta ferramenta é permite o carregamento tanto de mapas salvos ou a geração de mapas "instantâneos", criados na hora através de um algoritmo procedural de geração de mapas codificado em \textit{scripts lua}. Pode-se também escolher o número de vezes que um determinado mapa irá ser executado para a coleta das métricas. Os mapas simulados podem ser jogados através de jogos automatizados ou manuais, isto é, por comandos do jogador. 

\subsection{\textit{Map Builder} - Ferramenta auxiliar}

Esta ferramenta permite a criação de mapas de forma intuitiva e possui algumas ferramentas básicas de edição para o auxilio da produção de mapas. Ela permite a criação e posicionamento de inimigos e itens, pontos de entrada e saída do mapa, e especificação de passagem ou não por um determinado bloco. Todos os comandos e operações são visuais e facilmente vistas.

Possui também um editor textual capaz de identificar a sintaxe léxica da linguagem \textit{lua} e disponibilizando um dicionário de comandos entendidos pelas ferramentas para a criação dos mapas e a descrições de suas utilizações. Permite também que o algoritmos criado seja executado em tempo real e altere o mapa na área principal do programa, identificando erros de sintaxe ou de execução do \textit{scrip} caso ocorram e indicando sua linha.