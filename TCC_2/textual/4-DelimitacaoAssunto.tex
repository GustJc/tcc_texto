\chapter[Delimitação do Assunto]{Delimitação do Assunto}

Aqui serão representados o que este trabalho propõe-se a desenvolver e que resultados podem ser esperados através do uso da ferramenta a ser desenvolvida.

\section{Proposta}

Construir uma ferramenta capaz de medir a qualidade esperada de um mapa ou algoritmo procedural.
Propõe-se também a longo prazo, durante a segunda etapa do trabalho mostrar os benefícios que conteúdos procedurais trazem a jogos e uma comparação entre mapas feitos manualmente e automaticamente para demonstrar a diferença de qualidade entre os dois modelos de construção de mapas. 

\section{Ferramenta a ser desenvolvida}

A principal ferramenta a ser desenvolvida auferirá a qualidade de um mapa carregado através de um arquivo externo com o seu devido formato. A partir de entradas sobre os valores esperados pelo usuário da ferramenta, ela irá gerar saídas sobre a qualidade do mapa em relação aqueles quesitos especificados e aos diversos perfis de usuário.

A ferramenta também terá suporte à carga de algoritmos de geração de mapas através da sua codificação de arquivos externos no formato \textit{lua}, evitando o desgaste de ter que formatar cada um dos mapas para o formato da ferramenta e testá-los um a um. 

\subsection{Especificações}
O sistema deve ser capaz de obter um valor desejado para cada uma das métricas que estará testando a fim de garantir que o mapa estará a qualidade desejada para aquele propósito.

O sistema desenvolvido deverá ser capaz de ler de um arquivo externo, de acordo com uma formatação pré-estabelecida e a partir dele carregar um mapa e o posicionamento de inimigos e ítens.

A ferramenta deverá ser capaz de simular jogos, baseados em regras simples do gênero, para testar tais mapas.

Deverá analisar diversos jogos utilizando uma IA customizada para simular os diversos perfis de usuários encontrados.

Extrair métricas de $N$ jogos simulados e compilá-las em um parâmetro de qualidade, baseado nos valores esperados.

Deverá ser capaz carregar algoritmos de geração mapas escritos em \textit{lua} para que possam ser testado a qualidade do algoritmo como um todo  a partir dos mapas gerados por ele.

Deverá entender a padronização de ítens e inimigos em arquivos lua separados, para que possa escrever a criação dos mapas de forma mais clara e limpa.

\subsection{Escopo}

Está dentro do escopo a criação de uma ferramenta capaz de simular regras básicas de um jogo, e extrair métricas a partir dele. 

Está no escopo a criação de mapas baseados em arquivos ou algoritmos externos, dado que estejam no devido formato entendido pelo sistema.

Está no escopo a obtenção de uma métrica de qualidade para um devido mapa ou algoritmo testado.

\subsection{Não Escopo}

Não está no escopo deste projeto desenvolver um jogo completo, ou qualquer tipo de jogo. A simulação representado pela ferramenta, apesar de ser jogável, contém apenas os requisitos básicos para se testar uma amostra genérica dos atributos contidos em um jogo do gênero.

Não está no escopo a criação de algoritmos procedurais novos e otimizados, apesar de que serão utilizados algoritmos procedurais para gerar as entradas da ferramenta. Estes não serão completamente inventados ou otimizados e representarão diferentes tipos de geração procedural para dar a diversidade ao se testar a ferramenta e garantir que ela funciona apesar da diferença estrutural entre no que diz respeito à mapas.

Não está no escopo a criação de algoritmos de inteligência artificial completamente inéditos. Algoritmos de caminhos como \textit{A*} ou \textit{Depth First Search} serão utilizados com ligeiras alterações para se adaptarem ao sistema. 