\begin{resumo}[Abstract]
 \begin{otherlanguage*}{english}
  It will be analysed in this work the main characteristics of the roguelike genre gameplay and developed a tool capable of using a map data as input to perform an automated simulation of games to assure the quality based in several metrics and their expected values  for the map.

 \begin{comment}
   The creation of procedurally generated content in eletronic games over the last few years has been significantly increasing. The games of roguelike genre overstate themselfes as since their creation in the 80's, they have already been heavily using procedural generated maps and itens.
   
   Even though the heavy usage of procedural content, there is not a good metric today to measure in a precise way the quality of their maps. Usually having to adjust their generated maps based on several user tests and the gut feeling obteined after many runs.
   
   There is nothing wrong with the method, although if there was a tool capable of producing good metrics for the randômly generated maps in a realativelly fast way, that could accelarate the improvment of creation algorithms as the overall test time would be reduced. Meaning, you could get more imediate results in a shorter amount of time.
   
   In this project will be developed a system capable of obtaining a map and testing it through a series of metrics to assure que quality of the map in comparisson with a desired quality expected. As part of the second part of the project, a comparasing analysys will be made to state if randomly generated maps can have similar metrics as manual crafted ones and how real users feel about that, assuring the reability of the system in its results. 
\end{comment}
   \vspace{\onelineskip}
 
   \noindent 
   \textbf{Key-words}: procedural. games. algorithm. statistics.
 \end{otherlanguage*}
\end{resumo}
