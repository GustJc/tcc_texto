\chapter[Objetivos]{Objetivos}

Estão especificados aqui os objetivos gerais que mostram as metas de longo alcance que o trabalho procura atingir, de forma mais ampla. E os objetivos específicos que detalham e orientam a forma com que será voltado para completar os objetivos gerais.

\section{Objetivos Gerais}

Este trabalho procura desenvolver uma forma de análise de algoritmos procedurais voltados especificamente para jogos do gênero \textit{roguelike}. O objetivo principal é definir uma métrica para a qualidade de um determinado mapa, a ser obtida através de uma série de testes automatizados de jogabilidade deste mapa. Estes testes gerarão uma série de métricas e estatísticas que serão combinadas de modo a formar a métrica de qualidade desejada, permitindo a avaliação do mapa auto-gerado, em tempo real, de acordo com as características impostas pelo desenvolvedor do jogo. 

\section{Objetivos Especifícos}
\begin{description}
	\item [] \

	\begin{itemize}
		\item Explorar opções de API's gráficas para o sistema.
		\item Construir um sistema básico de simulação de jogos \textit{roguelike}.
		\item Criar uma inteligencia artificial para jogar automaticamente através de \textit{scripts}.
		\item Identificar e estabelecer perfis para possíveis AI(Inteligência Artificial) e melhor simular as métricas extraídas.
		\item Extrair métricas a partir de partidas jogadas automaticamente pela inteligência artificial.
		\item Normalizar e classificar as métricas extraídas de uma forma qualitativa através de acordo com parâmetros esperados.
		
	\end{itemize}
\end{description}
