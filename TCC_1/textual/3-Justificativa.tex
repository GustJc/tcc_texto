\chapter[Justificativa]{Justificativa}

Apesar da grande utilização de conteúdos procedurais em jogos do gênero \textit{roguelike}, ainda não foi estabelecida uma metodologia e um procedimento padrão para teste da qualidade dos algoritmos envolvidos e de suas saídas.

Parte da dificuldade de uma abordagem sistemática do problema é a dificuldade em mensurar certos fatores presentes em um jogo que não são facilmente medidos como, por exemplo, diversão, interesse, apreciação, etc. Porém há outras métricas que podem ser extraídas e compiladas para que juntas permitam a construção de um indicador associado à qualidade das saídas de um algoritmo procedural, segundo critérios pré-definidos pela equipe de desenvolvimento do jogo.

Uma ferramenta capaz de obter um mapa e realizar $N$ testes sobre ele, garantindo que o maior número de possibilidades possíveis sejam testadas em partidas simuladas automaticamente e capaz de extrair métricas significativas sobre a qualidade desejada pode servir de auxílio para a verificação e validação qualidade do conteúdo gerado sem que haja a longa e desgastante participação de usuários e ajustes nos algoritmos baseados apenas nos seus sentimentos e impressões pessoais.

Esta ferramenta não irá retirar o usuário do processo: ela servirá para selecionar e elencar  os mapas já previamente otimizados de acordo com as métricas geradas pela ferramenta. Esta abordagem permitirá alocar melhor o tempo e obter uma contribuição mais significativa dos usuários finais, levando a um refinamento e aperfeiçoamento da geração procedural de conteúdos do jogo.

