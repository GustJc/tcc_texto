\begin{resumo}
 Neste trabalho será analisado as principais características da jogabilidade do gênero \textit{roguelike} e desenvolvido uma ferramenta capaz de a partir de uma mapa como entrada, realizar testes utilizando jogos simulados automáticos para assegurar a sua qualidade com base diversas métricas e seus valores esperados para o mapa. 
 
 \begin{comment}
 
 Apesar de usarem tão fortemente conteúdos gerados proceduralmente, ainda hoje não é possível obter uma boa métrica de forma precisa quanto qualidade de seus mapas. Normalmente tendo a adequação dos seus mapas gerados através de muitos testes e apenas do sentimento obtido após muitas partidas.
 
 Não há nada de errado com esta forma, porém se houvesse uma ferramenta capaz de produzir métricas de análise para os mapas randômicamente gerados de forma relativamente rápida, isto poderia acelerar o processo de aperfeiçoamento dos algoritmos devido a redução em parte do tempo de testes. 
 
 Neste projeto, será desenvolvido um sistema capaz de obter um mapa e testa-lo através de uma série de métricas para assegurar a qualidade do mapa em relação a qualidade desejada. Na segunda parte do projeto será realizado uma analise comparativa para indicar se mapas proceduralmente gerados podem possuir metricas similares a mapas construidos manualmente e como usuários reais se sentem em relação as métricas obtidas, assegurando a confiança do sistema em seus resultados. 
\end{comment}
 \vspace{\onelineskip}
    
 \noindent
 \textbf{Palavras-chaves}: procedural. jogos. algoritmos. estatística.
\end{resumo}
